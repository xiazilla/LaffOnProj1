\documentclass[12pt]{article}

\usepackage{amssymb}
\usepackage{ifthen}
\usepackage[table]{xcolor}
\usepackage{minitoc}
\usepackage{array}

\definecolor{yellow}{cmyk}{0,0,1,0}
\renewcommand{\arraystretch}{1.4}
\newcommand{\R}{\mathbb{R}}

\usepackage{colortbl}

% Page size
\setlength{\oddsidemargin}{-0.5in}
\setlength{\evensidemargin}{-0.5in}
\setlength{\textheight}{10.25in}
\setlength{\textwidth}{7.0in}
\setlength{\topmargin}{-1.35in}

\renewcommand{\arraycolsep}{3pt}


\input color_flatex

\begin{document}
\pagestyle{empty}


\resetsteps % reset all definitions

% Insert output from Spark webpage below


\resetsteps      % Reset all the commands to create a blank worksheet  

% Define the operation to be computed


\resetsteps      % Reset all the commands to create a blank worksheet  

% Define the operation to be computed

\renewcommand{\operation}{ B = L B }

\renewcommand{\routinename}{\operation}

% Step 1a: Precondition 

\renewcommand{\precondition}{
	B = \widehat{B}
}

% Step 1b: Postcondition 

\renewcommand{\postcondition}{ 
	B = L \widehat B
}

% Step 2: Invariant 
% Note: Right-hand side of equalities must be updated appropriately

\renewcommand{\invariant}{
	\left(\begin{array}{c}
		B_T \\ \whline
		B_B 
	\end{array}\right)  = 
	\left(\begin{array}{c}
		\widehat B_T \\ \whline 
		L_{BR} \widehat B_B 
	\end{array}\right)
}

% Step 3: Loop-guard 

\renewcommand{\guard}{
	m( L_{BR} ) < m( L )
}

% Step 4: Initialize 

\renewcommand{\partitionings}{
	$
	L \rightarrow
	\left(\begin{array}{c I c} 
	L_{TL} & L_{TR} \\ \whline
	L_{BL} & L_{BR} 
	\end{array}\right) 
	$
	,
	$
	B \rightarrow
	\left(\begin{array}{c}
	B_{T} \\ \whline 
	B_{B}
	\end{array}\right)
	$
}

\renewcommand{\partitionsizes}{
	$ L_{BR} $ is $ 0 \times 0 $,
	$ B_B $ has $ 0 $ rows
}

% Step 5a: Repartition the operands 

\renewcommand{\repartitionings}{
	$  \left(\begin{array}{c I c}
	L_{TL} & L_{TR} \\ \whline 
	L_{BL} & L_{BR}
	\end{array}\right) 
	\rightarrow
	\left(\begin{array}{c c I c}
	L_{00} & l_{01} & L_{02} \\  
	l_{10}^T & \lambda_{11} & l_{12}^T \\ \whline 
	L_{20} & l_{21} & L_{22}
	\end{array}\right) 
	$
	,
	$  \left(\begin{array}{c}
	B_T \\ \whline
	B_B 
	\end{array}\right) 
	\rightarrow
	\left(\begin{array}{c}
	B_0 \\  
	b_1^T \\ \whline 
	B_2
	\end{array}\right)
	$
}

\renewcommand{\repartitionsizes}{
	$ \lambda_{11} $ is $ 1 \times 1 $,
	$ b_1 $ has $ 1 $ row}

% Step 5b: Move the double lines 

\renewcommand{\moveboundaries}{
	$  \left(\begin{array}{c I c}
	L_{TL} & L_{TR} \\ \whline 
	L_{BL} & L_{BR}
	\end{array}\right) 
	\leftarrow
	\left(\begin{array}{c I c c}
	L_{00} & l_{01} & L_{02} \\ \whline 
	l_{10}^T & \lambda_{11} & l_{12}^T \\  
	L_{20} & l_{21} & L_{22}
	\end{array}\right) 
	$
	,
	$  \left(\begin{array}{c}
	B_T \\ \whline
	B_B 
	\end{array}\right) 
	\leftarrow
	\left(\begin{array}{c}
	B_0 \\ \whline 
	b_1^T \\  
	B_2
	\end{array}\right) 
	$
}

% Step 6: State before update
% Note: The below needs editing consistent with loop-invariant!!!

\renewcommand{\beforeupdate}{$ 
	\left(\begin{array}{c}
	B_0 \\ \whline 
	b_1^T \\  
	B_2
	\end{array}\right) 
	=
	\left(\begin{array}{c}
	\widehat B_0 \\ \whline 
	\widehat b_1^T \\  
	L_{22} \widehat B_2
	\end{array}\right) 
$}


% Step 7: State after update
% Note: The below needs editing consistent with loop-invariant!!!

\renewcommand{\afterupdate}{$ 
		\left(\begin{array}{c}
		B_0 \\ \whline 
		b_1^T \\  
		B_2
		\end{array}\right) 
		=
		\left(\begin{array}{c}
		\widehat B_0 \\ \whline 
		\lambda_{11} \widehat b_1^T \\  
		l_{21} b_1^T + L_{22} \widehat B_2
		\end{array}\right) 
	$}


% Step 8: Insert the updates required to change the 
%         state from that given in Step 6 to that given in Step 7
% Note: The below needs editing!!!

\renewcommand{\update}{
	$
	\begin{array}{l}          % do not delete this line 
	B_2 := l_{21} b_1^T \\
	b_1^T := \lambda_{11} b_1^T 
	\end{array}               % do not delete this line 
	$
}








\begin{center}
	\FlaWorksheet
\end{center}

\newpage

\begin{figure}[p]
\begin{center}
	\FlaWorksheetZero
\end{center}
\end{figure}

\begin{figure}[p]
\begin{center}
	\FlaWorksheetOne
\end{center}
\end{figure}

\begin{figure}[p]
\begin{center}
	\FlaWorksheetTwo
\end{center}
\end{figure}

\begin{figure}[p]
\begin{center}
	\FlaWorksheetThree
\end{center}
\end{figure}

\begin{figure}[p]
	\begin{center}
	\FlaWorksheetFour
\end{center}
\end{figure}

\begin{figure}[p]
	\begin{center}
	\FlaWorksheetFive
\end{center}
\end{figure}

\begin{figure}[p]
	\begin{center}
	\FlaWorksheetSix
\end{center}
\end{figure}

\begin{figure}[p]
	\begin{center}
	\FlaWorksheetSeven
\end{center}
\end{figure}

\begin{figure}[p]
	\begin{center}
	\FlaWorksheetEight
\end{center}
\end{figure}

\begin{figure}[p]
	\begin{center}
	\FlaWorksheetNine
\end{center}
\end{figure}

\begin{figure}[p]
\begin{center}
	\FlaAlgorithm
\end{center}
\end{figure}

\end{document}